
\documentclass[a4paper,11pt]{report}
%
%--------------------   start of the 'preamble'
%
\usepackage{graphicx,amssymb,amstext,amsmath}
\usepackage{enumerate}
\usepackage{hyperref}
\usepackage[spanish]{babel}
\usepackage[utf8]{inputenc}
\hypersetup{
    colorlinks,
    citecolor=black,
    filecolor=black,
    linkcolor=black,
    urlcolor=black
}
%
%%    homebrew commands -- to save typing
\newcommand\etc{\textsl{etc}}
\newcommand\eg{\textsl{eg.}\ }
\newcommand\etal{\textsl{et al.}}
\newcommand\Quote[1]{\lq\textsl{#1}\rq}
\newcommand\fr[2]{{\textstyle\frac{#1}{#2}}}
\newcommand\miktex{\textsl{MikTeX}}
\newcommand\comp{\textsl{The Companion}}
\newcommand\nss{\textsl{Not so Short}}
%
%---------------------   end of the 'preamble'
%
\begin{document}
%-----------------------------------------------------------
\title{Metodologías de desarrollo}
\author{Patricio Pérez\\
        Sebastián Rocha\\
        Natalia Tarifeño}
\maketitle
%-----------------------------------------------------------
\tableofcontents
Probando...
\begin{center}
    \begin{tabular}{ | p{7cm} | p{7cm} |}
    \hline
    Metodología Tradicional & Metodología Ágil \\ \hline
    Es iterativo y asigna tareas de una estructurada. & Es flexible a la entrada de tareas y procesos, la priorización y supervisión del equipo de trabajo \\ \hline
    Es un conjunto de metodologías que se adaptan a las necesidad, completamente estructurado. & No es sólo un método, es un sistema de mejora en el desarrollo de un proyecto. \\ \hline
    Logra encontrar errores en forma temprana, focalizando a los desarrolladores en producir resultados. & Se centra principalmente en la calidad en realizar los proyecto, más que en su rapidez. \\ \hline
    Se desarrolla mediante 4 fases:
    \begin{enumerate}
        \item Inicio
        \item Elaboración
        \item Desarrollo
        \item Cierre
    \end{enumerate} & Consta de 4 fases:
    \begin{enumerate}
        \item Análisis
        \item Desarrollo
        \item Pruebas
        \item Deploy
    \end{enumerate} \\ \hline
    Se basa en las mejores prácticas que se han implementado y se han probado en el campo. & Se basa en hacer lo justo y necesario pero de buena manera definiendo así que va primero com para prevenir el exceso de información innecesaria. \\ \hline
    Por su grado de complejidad y su metodo rustico no es muy agradable o adecuado. & Es fácil de usar e utilizar, actualizar y es amigable con el entorno que lo utilizara. \\ \hline
    Su propósito principal es asegurar una producción de software de alta calidad que cumple con los requisitos del usuario. & Su propósito es simplificar la comunicación agilizando y evitando errores producidos por la falta de información. \\
    \hline
    \end{tabular}
\end{center}
%-----------------------------------------------------------
\end{document}
